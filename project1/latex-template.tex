% Préambule : Configuration du document
\documentclass[12pt,a4paper]{report}  % Type de document, taille de police et format papier

% Packages essentiels
\usepackage[utf8]{inputenc}           % Pour les caractères accentués
\usepackage[T1]{fontenc}              % Pour la police
\usepackage[french]{babel}            % Pour la langue française
\usepackage{graphicx}                 % Pour insérer des images
\usepackage{amsmath,amsthm,amssymb}   % Pour les formules mathématiques
\usepackage{hyperref}                 % Pour les liens hypertextes
\usepackage{listings}                 % Pour insérer du code
\usepackage{xcolor}                   % Pour la couleur
\usepackage{geometry}                 % Pour les marges
\usepackage{fancyhdr}                 % Pour les en-têtes et pieds de page
\usepackage{titlesec}                 % Pour personnaliser les titres
\usepackage{tocloft}                  % Pour personnaliser la table des matières

% Configuration des marges
\geometry{
    top=2.5cm,
    bottom=2.5cm,
    left=2.5cm,
    right=2.5cm
}

% Configuration des en-têtes et pieds de page
\pagestyle{fancy}
\fancyhf{}
\fancyhead[L]{\leftmark}
\fancyhead[R]{\thepage}
\renewcommand{\headrulewidth}{0.4pt}

% Configuration pour le code
\lstset{
    frame=single,
    breaklines=true,
    postbreak=\raisebox{0ex}[0ex][0ex]{\ensuremath{\color{red}\hookrightarrow\space}},
    numbers=left,
    numberstyle=\tiny\color{gray},
    basicstyle=\ttfamily\footnotesize,
    keywordstyle=\color{blue},
    commentstyle=\color{green!60!black},
    stringstyle=\color{orange}
}

% Informations du document
\title{{\Huge Titre du Rapport}\\[0.5cm]
       {\large Sous-titre}}
\author{Votre Nom\\
        \texttt{votre.email@domain.com}}
\date{\today}

% Début du document
\begin{document}

% Page de titre
\maketitle

% Page de déclaration/copyright (optionnel)
\clearpage
\thispagestyle{empty}
\vspace*{\fill}
\begin{center}
    \textcopyright{} 2024 - Tous droits réservés\\
    [Nom de votre organisation]
\end{center}
\vspace*{\fill}
\clearpage

% Table des matières
\tableofcontents
\clearpage

% Liste des figures (si nécessaire)
\listoffigures
\clearpage

% Liste des tableaux (si nécessaire)
\listoftables
\clearpage

% Introduction
\chapter{Introduction}
\section{Contexte}
Écrivez votre contexte ici.

\section{Objectifs}
Listez vos objectifs ici.

% Corps du rapport
\chapter{État de l'art}
\section{Section 1}
Voici un exemple de texte avec une citation\cite{reference1}.

% Comment insérer une image
\begin{figure}[htbp]
    \centering
    \includegraphics[width=0.8\textwidth]{chemin/vers/image}
    \caption{Description de l'image}
    \label{fig:exemple}
\end{figure}

% Comment insérer un tableau
\begin{table}[htbp]
    \centering
    \begin{tabular}{|c|c|c|}
        \hline
        Colonne 1 & Colonne 2 & Colonne 3 \\
        \hline
        Valeur 1 & Valeur 2 & Valeur 3 \\
        \hline
    \end{tabular}
    \caption{Description du tableau}
    \label{tab:exemple}
\end{table}

% Comment insérer une équation
\begin{equation}
    E = mc^2
    \label{eq:einstein}
\end{equation}

% Comment insérer du code
\begin{lstlisting}[language=Python, caption=Exemple de code Python]
def hello_world():
    print("Hello, World!")
\end{lstlisting}

% Conclusion
\chapter{Conclusion}
Écrivez votre conclusion ici.

% Bibliographie
\bibliographystyle{plain}
\bibliography{references}  % Fichier references.bib

% Annexes
\appendix
\chapter{Annexe A}
Contenu de l'annexe A.

\end{document}
